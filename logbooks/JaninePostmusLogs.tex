% Options for packages loaded elsewhere
\PassOptionsToPackage{unicode}{hyperref}
\PassOptionsToPackage{hyphens}{url}
%
\documentclass[
]{article}
\usepackage{amsmath,amssymb}
\usepackage{iftex}
\ifPDFTeX
  \usepackage[T1]{fontenc}
  \usepackage[utf8]{inputenc}
  \usepackage{textcomp} % provide euro and other symbols
\else % if luatex or xetex
  \usepackage{unicode-math} % this also loads fontspec
  \defaultfontfeatures{Scale=MatchLowercase}
  \defaultfontfeatures[\rmfamily]{Ligatures=TeX,Scale=1}
\fi
\usepackage{lmodern}
\ifPDFTeX\else
  % xetex/luatex font selection
\fi
% Use upquote if available, for straight quotes in verbatim environments
\IfFileExists{upquote.sty}{\usepackage{upquote}}{}
\IfFileExists{microtype.sty}{% use microtype if available
  \usepackage[]{microtype}
  \UseMicrotypeSet[protrusion]{basicmath} % disable protrusion for tt fonts
}{}
\makeatletter
\@ifundefined{KOMAClassName}{% if non-KOMA class
  \IfFileExists{parskip.sty}{%
    \usepackage{parskip}
  }{% else
    \setlength{\parindent}{0pt}
    \setlength{\parskip}{6pt plus 2pt minus 1pt}}
}{% if KOMA class
  \KOMAoptions{parskip=half}}
\makeatother
\usepackage{xcolor}
\usepackage[margin=1in]{geometry}
\usepackage{color}
\usepackage{fancyvrb}
\newcommand{\VerbBar}{|}
\newcommand{\VERB}{\Verb[commandchars=\\\{\}]}
\DefineVerbatimEnvironment{Highlighting}{Verbatim}{commandchars=\\\{\}}
% Add ',fontsize=\small' for more characters per line
\usepackage{framed}
\definecolor{shadecolor}{RGB}{248,248,248}
\newenvironment{Shaded}{\begin{snugshade}}{\end{snugshade}}
\newcommand{\AlertTok}[1]{\textcolor[rgb]{0.94,0.16,0.16}{#1}}
\newcommand{\AnnotationTok}[1]{\textcolor[rgb]{0.56,0.35,0.01}{\textbf{\textit{#1}}}}
\newcommand{\AttributeTok}[1]{\textcolor[rgb]{0.13,0.29,0.53}{#1}}
\newcommand{\BaseNTok}[1]{\textcolor[rgb]{0.00,0.00,0.81}{#1}}
\newcommand{\BuiltInTok}[1]{#1}
\newcommand{\CharTok}[1]{\textcolor[rgb]{0.31,0.60,0.02}{#1}}
\newcommand{\CommentTok}[1]{\textcolor[rgb]{0.56,0.35,0.01}{\textit{#1}}}
\newcommand{\CommentVarTok}[1]{\textcolor[rgb]{0.56,0.35,0.01}{\textbf{\textit{#1}}}}
\newcommand{\ConstantTok}[1]{\textcolor[rgb]{0.56,0.35,0.01}{#1}}
\newcommand{\ControlFlowTok}[1]{\textcolor[rgb]{0.13,0.29,0.53}{\textbf{#1}}}
\newcommand{\DataTypeTok}[1]{\textcolor[rgb]{0.13,0.29,0.53}{#1}}
\newcommand{\DecValTok}[1]{\textcolor[rgb]{0.00,0.00,0.81}{#1}}
\newcommand{\DocumentationTok}[1]{\textcolor[rgb]{0.56,0.35,0.01}{\textbf{\textit{#1}}}}
\newcommand{\ErrorTok}[1]{\textcolor[rgb]{0.64,0.00,0.00}{\textbf{#1}}}
\newcommand{\ExtensionTok}[1]{#1}
\newcommand{\FloatTok}[1]{\textcolor[rgb]{0.00,0.00,0.81}{#1}}
\newcommand{\FunctionTok}[1]{\textcolor[rgb]{0.13,0.29,0.53}{\textbf{#1}}}
\newcommand{\ImportTok}[1]{#1}
\newcommand{\InformationTok}[1]{\textcolor[rgb]{0.56,0.35,0.01}{\textbf{\textit{#1}}}}
\newcommand{\KeywordTok}[1]{\textcolor[rgb]{0.13,0.29,0.53}{\textbf{#1}}}
\newcommand{\NormalTok}[1]{#1}
\newcommand{\OperatorTok}[1]{\textcolor[rgb]{0.81,0.36,0.00}{\textbf{#1}}}
\newcommand{\OtherTok}[1]{\textcolor[rgb]{0.56,0.35,0.01}{#1}}
\newcommand{\PreprocessorTok}[1]{\textcolor[rgb]{0.56,0.35,0.01}{\textit{#1}}}
\newcommand{\RegionMarkerTok}[1]{#1}
\newcommand{\SpecialCharTok}[1]{\textcolor[rgb]{0.81,0.36,0.00}{\textbf{#1}}}
\newcommand{\SpecialStringTok}[1]{\textcolor[rgb]{0.31,0.60,0.02}{#1}}
\newcommand{\StringTok}[1]{\textcolor[rgb]{0.31,0.60,0.02}{#1}}
\newcommand{\VariableTok}[1]{\textcolor[rgb]{0.00,0.00,0.00}{#1}}
\newcommand{\VerbatimStringTok}[1]{\textcolor[rgb]{0.31,0.60,0.02}{#1}}
\newcommand{\WarningTok}[1]{\textcolor[rgb]{0.56,0.35,0.01}{\textbf{\textit{#1}}}}
\usepackage{longtable,booktabs,array}
\usepackage{calc} % for calculating minipage widths
% Correct order of tables after \paragraph or \subparagraph
\usepackage{etoolbox}
\makeatletter
\patchcmd\longtable{\par}{\if@noskipsec\mbox{}\fi\par}{}{}
\makeatother
% Allow footnotes in longtable head/foot
\IfFileExists{footnotehyper.sty}{\usepackage{footnotehyper}}{\usepackage{footnote}}
\makesavenoteenv{longtable}
\usepackage{graphicx}
\makeatletter
\def\maxwidth{\ifdim\Gin@nat@width>\linewidth\linewidth\else\Gin@nat@width\fi}
\def\maxheight{\ifdim\Gin@nat@height>\textheight\textheight\else\Gin@nat@height\fi}
\makeatother
% Scale images if necessary, so that they will not overflow the page
% margins by default, and it is still possible to overwrite the defaults
% using explicit options in \includegraphics[width, height, ...]{}
\setkeys{Gin}{width=\maxwidth,height=\maxheight,keepaspectratio}
% Set default figure placement to htbp
\makeatletter
\def\fps@figure{htbp}
\makeatother
\setlength{\emergencystretch}{3em} % prevent overfull lines
\providecommand{\tightlist}{%
  \setlength{\itemsep}{0pt}\setlength{\parskip}{0pt}}
\setcounter{secnumdepth}{-\maxdimen} % remove section numbering
\ifLuaTeX
  \usepackage{selnolig}  % disable illegal ligatures
\fi
\usepackage{bookmark}
\IfFileExists{xurl.sty}{\usepackage{xurl}}{} % add URL line breaks if available
\urlstyle{same}
\hypersetup{
  pdftitle={Logbook Janine Postmus},
  pdfauthor={Janine Postmus},
  hidelinks,
  pdfcreator={LaTeX via pandoc}}

\title{Logbook Janine Postmus}
\author{Janine Postmus}
\date{2024-09-10}

\begin{document}
\maketitle

\subsubsection{Inleiding genomics}\label{inleiding-genomics}

In dit logboek zal ik bijhouden wat ik voor het project Genomics \&
Transcriptomics, leerjaar 2 kwartaal 1, allemaal uitvoer. Hierbij
onderbouw ik mijn keuzes met waarom ik iets doe hoe ik dat heb gedaan.
Ik maak gebruik van de wat, hoe en waarom methode. Wat heb ik gedaan?
Hoe heb ik dit gedaan en waarom heb ik dit gedaan? Ook zal ik hier
resultaten van tonen.

\#\#09-09-2024\\
Nadat we de uitleg hebben gekregen hebben we groepen gevormd. Mijn groep
is samen met Ramon, Jasper en Iris. We hebben vervolgens een begin
gemaakt met het zoeken naar een wetenschappelijk artikel. Dit hebben we
gedaan via de volgende link:
\href{https://www.ncbi.nlm.nih.gov/gds/?term=(\%22expression+profiling+by+high+throughput+sequencing\%22\%5BDataSet+Type\%5D)+AND+\%22genome+variation+profiling+by+high+throughput+sequencing\%22\%5BDataSet+Type\%5D}{Wetenschappelijke
artikelen} ~ We hebben afgesproken dat we allemaal 1 artikel uitzoeken
en uiteindelijk op basis van Marcel zijn oordeel een definieve keus
maken. Marcel controleert de artikelen of deze voldoende informatie
bevatten zodat het voor ons te repliceren is. De keuzes waren:\\
- MLL-leukemie-inductie door t(9;11)-chromosomale translocatie in
menselijke hematopoietische stamcellen met behulp van genoombewerking.
Gekozen door Iris.\\
- De behandeling van Entinostat op blaaskanker in de muis. Gekozen door
Jasper.\\
- IGF2BP1 geïnduceert neuroblastoom dat oncogenexpressie bevordert.
Gekozen door Ramon.\\
- Het DNA-methylatielandschap van de progressie van de ziekte van
glioblastoom. Gekozen door Janine.\\

\#\#10-09-2024\\
De experimenten van Janine en Iris zijn afgekeurd wegens ontbrekende of
onvoldoende data. Vervolgens hebben we als groep gekozen voor Jasper
zijn artikel.\\
We hebben gekozen voor een onderzoek met de behandeling van Entinostat
op blaaskanker. Het onderwerp interesseert mij heel erg omdat ik kanker
een interessant onderwerp vind. De entinostat werkt heel erg op
genexpressie en het immuunsysteem. Ik vind genexpressie erg interessant
omdat dit heel veel bepaald in ons lichaam.\\
Onze onderzoeksvraag is: Hoe kunnen de mutaties die we vinden
blaaskanker veroorzaken?\\

\#\#11-09-2024\\
We hebben een taakverdeling gemaakt voor het plan van aanpak.\\
Ik ga het onderzoekoverzicht, de achtergrond ervan en een globale
theoretische achtergrond presenteren tijdens de presentatie.\\
Iris gaat de Gantt chart maken.\\
Jasper gaat de workflow maken.\\
Ramon zet het plan van aanpak en de presentatie in elkaar. ~ Verder
hadden we vandaag statistiek les dus geen tijd meer gehad voor dit
project.\\

\#\#12-09-2024\\
Ik heb het artikel doorgelezen en de kernwoorden eruit gehaald zodat ik
hierop verder kon bouwen aan de theoretische achtergrond. De volgende
kernwoorden heb ik gebruikt:\\
- Immuuncheckpointremmers (ICI's)\\
- Entinostat\\
- Histondeacetylase remmers (HDAC remmers)\\
- Immuunreactie op tumor-neo antigenen\\

Ik heb vanuit het artikel de kopjes Abstract en Introduction hiervoor
gebruikt. Hier staat dusdanig veel informatie in over het onderzoek dat
hier voor het plan van aanpak voldoende uit te halen valt. Ik heb
gebruik gemaakt van online resourches en voor het vertalen van het
artikel heb ik gebruik gemaakt van copilot. Voor de bronnen en de
verwerking van de kernwoorden kan gekeken worden in ons Plan van Aanpak.
Dit is te vinden in onze
\href{https://github.com/RamonReilman/GenomicsTranscriptomics}{Github}
onder Plan van aanpak.\\

\#\#13-09-2024\\
Ik heb de theorie aangevuld met de laatste dingen. In de pva staat die
onder alinea `Entinostat en zijn werking'.\\
Smiddags is er een les gevolgd over hoe we data kunnen downloaden op de
schoolserver. Hier heeft Ramon meegedaan met de les en het voor onze
data uitgevoerd. Ik heb voor mijzelf aantekeningen gemaakt.

\#\#15-09-2024\\
Naar aanleiding van feedback vanuit de groep heb ik een deel van de
theorie aangepast en aangevuld. Er klopte iets niet over de
projectopzet, ik heb dit gecorrigeerd en vervolgens gemeld aan de groep
zodat de presentatie afgemaakt kon worden.\\
De presentatie is afgemaakt door Ramon. Omdat ik met een gebroken enkel
in huis vast zit, gaat Ramon mijn deel van de presentatie doen en gaan
we maandag afspraken maken over de verdere uitvoering van het project.

\#\#17-09-2024\\
Ik heb de tool snpEff gekregen/gekozen als onderdeel van het project.
Omdat ik de komende weken minder op school aanwezig kan zijn hebben we
deze tool aan mij toebesteed omdat deze tool pas redelijk aan het einde
van het proces nodig is. Hierdoor krijg ik meer tijd om processen en
dingen te onderzoeken voordat we de tool daadwerkelijk nodig zijn.

De verdeling van de tools is alsvolgt:\\
- BWA - Mapping --\textgreater{} Jasper\\
- Picard - Duplicates --\textgreater{} Iris\\
- Strelka - Variant calling --\textgreater{} Ramon\\
- SnpEff - Variant annotation --\textgreater{} Janine\\

De volgende tools hebben we allemaal zelf geleerd en uitgevoerd:\\
- Seqkit -\textgreater{} Making testdata ~ - Fastqc -\textgreater{}
Making data quality rapports\\
- Multiqc -\textgreater{} Making data quality rapports from multiple
fastqc rapports\\
- Trimmomatic - Trimming data to enhance data quality\\

Om te leren hoe we moeten omgaan met de data en testdata heb ik met
behulp van de tool seqkit testdata gemaakt voor ons project. Ik heb de
testdata gemaakt op basis van onze echte data.\\
Ik heb dit eerst uitgevoerd op 1 bestand om te leren hoe het werkt.
Omdat onze dataset uit heel veel bestanden bestaat, is het noodzakelijk
dat ik leer hoe ik het kan uitvoeren op alle bestanden. Ik weet dat ik
de find functie moet gebruiken icm de tools fastqc en parallel, hoe ik
het moet gebruiken weet ik nog niet. Verder hebben we afgesproken dat ik
alle stappen van het preprocessing traject, dus de fastqc, multiqc en
trimmomatic zelf ga uitvoeren met testdata om goed te begrijpen wat we
doen.

Gebruikte commandline voor seqkit om testdata te maken van 1 fastq file:

\begin{Shaded}
\begin{Highlighting}[]
\ExtensionTok{Seqkit}\NormalTok{ head }\AttributeTok{{-}n}\NormalTok{ 10000 /.x.fastq }\AttributeTok{{-}o}\NormalTok{ /.x.fastq}
\end{Highlighting}
\end{Shaded}

Hierin is /. x.fastq een inputfile, hierbij moet het absolute pad naar
de map gebruikt worden.\\
-o /.x.fastq is het outputfile, ook hier moet het absolute pad naar de
map gebruikt worden.\\

Vervolgens moet er een fastqc rapport gemaakt worden van alle files in
de map. Fastqc is een tool die ervoor zorgt dat de data overzichtelijk
in een quality rapport komt te staan. Hier worden de volgende onderdelen
op kwaliteit beoordeeld:\\
- Basic Statistics\\
- Per sequence counts\\
- Per base sequence quality\\
- Per sequence quality scores\\
- Per base sequence content\\
- Per sequence GC content\\
- Per base N content\\
- Sequence Duplication Levels\\
- Overrepresented Sequences\\
- Adapter Content\\

Fastqc uitvoeren op meerdere bestanden in 1 keer.

\begin{Shaded}
\begin{Highlighting}[]
\FunctionTok{find}\NormalTok{ /students/2024{-}2025/Thema05/BlaasKanker/fastq }\AttributeTok{{-}name} \StringTok{"*.fastq"} \KeywordTok{|}\ExtensionTok{parallel}\NormalTok{ fastqc }\AttributeTok{{-}o}\NormalTok{ /students/2024{-}2025/Thema05/BlaasKanker/oefenengroep/testdataJanine/MultiqcJanine/\{\}}
\end{Highlighting}
\end{Shaded}

Find wordt gebruikt om de map met files te vinden en hiervan een lijst
te maken. Path to file staat voor het volledige pad naar het bestand.
Het invoer bestand was in dit geval onze map met testdata. De output is
in een persoonlijke map gekomen waarin we alle 4 zelf kunnen testen. De
\{\} wordt gebruikt om aan te geven dat hier de uitvoerbestanden
inkomen.\\

\#\#18-09-2024\\
Omdat er met veel data gewerkt wordt heb je dus veel bestanden waarvan
een fastqc rapport gemaakt wordt. Omdat het heel moeilijk, zo niet
onmogelijk, is om al deze bestanden met het oog los van elkaar te
beoordelen gebruiken we hiervoor de tool multiqc. Ik heb uitgezocht hoe
ik het multiqc rapport moet maken. Via verschillende bronnen en forums
op internet heb ik de goede commandline geproduceerd en vervolgens
uitgevoerd.\\

\begin{Shaded}
\begin{Highlighting}[]
\FunctionTok{find}\NormalTok{ find /students/2024{-}2025/Thema05/BlaasKanker/fastq }\AttributeTok{{-}name} \StringTok{"*.fastq"} \KeywordTok{|}\ExtensionTok{parallel}\NormalTok{ multiqc }\AttributeTok{{-}o}\NormalTok{ /students/2024{-}2025/Thema05/BlaasKanker/oefenengroep/testdataJanine/MultiqcJanine/\{\}}
\end{Highlighting}
\end{Shaded}

\#\#20-09-2024\\
Om beter te begrijpen wat al deze kwaliteitscontroles, van de fastqc en
multiqc, nu eigenlijk zeggen heb ik hier een verslag van gemaakt. Dit
verslag staat in onze teams groep. ``2.1.2 Genomics \& transcriptomics
Ramon Reijlman, Janine Postmus, Jasper Jonker en Iris Ineke'' met de
titel ``Verslag\_multiqc\_rapport.docx''.\\

Ik vind het belangrijk om het op deze manier te hebben gedaan omdat ik
op deze manier sneller inzicht kan creeëren voor mijzelf hoe zo'n
rapport werkt en welke onderdelen het belangrijkste zijn. Ik zal hier de
conclusie neerzetten die ook in het verslag staat. Dit is op basis van
de getrimde data welke we op dit moment hebben.\\

Conclusie:\\
Onze data is op de meeste vlakken goed/zeer goed. Het enige wat opvalt
is dat er een relatief hoog duplicatieniveau is. Dit kan meerdere
oorzaken hebben. De meest waarschijnlijke reden in deze data is dat er
sprake is van PCR overamplificatie of van zeer expressieve genen in de
RNA-sequentie data. Omdat er gebruik is gemaakt van Entinostat wat de
expressie van genen stimuleert is dit een optie. Omdat we de data niet
zelf hebben geproduceerd, kunnen we hier nooit helemaal met zekerheid
iets over zeggen.

\#\#22-09-2024\\
Omdat het trimmen van de data nog niet goed is gegaan hebben wij dit
allemaal opnieuw uitgevoerd en uitgezocht wat de goede commandline is.
Ik heb dit gedaan door eerst te kijken naar wat trimmomatic is voor
programma en de documentatie goed door te nemen. Ik heb hiervoor de
officiële documentatie gelezen van trimmomatic via:
\href{http://www.usadellab.org/cms/uploads/supplementary/Trimmomatic/TrimmomaticManual_V0.32.pdf}{documentatie
trimmomatic}\\
Verder heb ik deze video op youtube
gekeken:\href{https://www.youtube.com/watch?v=Op3W5TEej3k}{bioinformatics
trimmomatic}\\

Ik heb in de documentatie vooral gekeken naar de opties die je met
trimmomatic kan meegeven en welke gunstig zouden zijn voor onze data. Ik
denk dat de opties ILUMACLIP, MINLEN, LEADING en TRAILING voor ons
project gunstig zijn.\\
- ILUMACLIP zorgt ervoor dat als er adapters zijn toegevoegd aan de data
dat deze gevonden en verwijderd worden. Adaptersequenties kunnen de
kwaliteit van de data beïnvloeden en ervoor zorgen dat er foutieve
resultaten in de analyse zitten. Hierdoor krijg je betrouwbaardere
informatie.\\
- MINLEN is een optie waarbij je kan aangeven dat er een het aantal
reads, dat onder de gespecificeerde minimumlengte zit, verwijderd
wordt.\\
- LEADING is een optie welke de basen aan de voorkant van de read, welke
meestal een slechte kwaliteit hebben, verwijderd wordt.\\
- TRAILING is een optie welke de basen aan de achterkant van de read,
welke meestal een slechte kwaliteit hebben, verwijderd wordt.\\

\#\#23-09-2024\\
Verder gegaan met trimmomatic, het zoeken naar de goede instellingen
zodat het fastq rapport zoveel mogelijk groen kleurt. Dit is belangrijk
om zo te kunnen kijken of de data nog verder geperfectioneerd kan
worden. Hoe meer groen de dataset is hoe beter deze gebruikt kan worden
voor de volgende stap in het proces, de mapping.\\

Ik heb het command voor trimmomatic hier neergezet om het zo makkelijker
aan te kunnen passen om makkelijker met de settings te kunnen
experimenteren, dit is ook het uiteindelijke command geworden.\\

\begin{Shaded}
\begin{Highlighting}[]
\FunctionTok{cat}\NormalTok{ /students/2024{-}2025/Thema05/BlaasKanker/SRR\_Acc\_List.txt }\KeywordTok{|} \DataTypeTok{\textbackslash{}}
    \ExtensionTok{parallel} \StringTok{\textquotesingle{}TrimmomaticPE {-}threads 10\textquotesingle{}} \DataTypeTok{\textbackslash{}}
            \StringTok{\textquotesingle{}/students/2024{-}2025/Thema05/BlaasKanker/outputs/fastq/\{\}\_1.fastq\textquotesingle{}} \DataTypeTok{\textbackslash{}}
            \StringTok{\textquotesingle{}/students/2024{-}2025/Thema05/BlaasKanker/outputs/fastq/\{\}\_2.fastq\textquotesingle{}} \DataTypeTok{\textbackslash{}}
            \StringTok{\textquotesingle{}/students/2024{-}2025/Thema05/BlaasKanker/oefenengroep/testdataJanine/trimdata/paird/\{\}paired\_1.fastq\textquotesingle{}} \DataTypeTok{\textbackslash{}}
            \StringTok{\textquotesingle{}/students/2024{-}2025/Thema05/BlaasKanker/oefenengroep/testdataJanine/trimdata/unpaird/\{\}unpaired\_1.fastq\textquotesingle{}} \DataTypeTok{\textbackslash{}}
            \StringTok{\textquotesingle{}/students/2024{-}2025/Thema05/BlaasKanker/oefenengroep/testdataJanine/trimdata/paird/\{\}paired\_2.fastq\textquotesingle{}} \DataTypeTok{\textbackslash{}}
            \StringTok{\textquotesingle{}/students/2024{-}2025/Thema05/BlaasKanker/oefenengroep/testdataJanine/trimdata/unpaird/\{\}unpaired\_2.fastq\textquotesingle{}} \DataTypeTok{\textbackslash{}}
            \StringTok{\textquotesingle{}LEADING:3\textquotesingle{}} \DataTypeTok{\textbackslash{}}
            \StringTok{\textquotesingle{}TRAILING:3\textquotesingle{}} \DataTypeTok{\textbackslash{}}
            \StringTok{\textquotesingle{}SLIDINGWINDOW:4:20\textquotesingle{}} \DataTypeTok{\textbackslash{}}
            \StringTok{\textquotesingle{}MINLEN:40\textquotesingle{}}
\end{Highlighting}
\end{Shaded}

\#\#24-09-2024\\
Het trimmen van de data heb ik afgemaakt, het zoeken naar de goede
settings om hier vervolgens fastqc op los te laten en een multiqc
rapport maken zodat ik dit kan vergelijken met het multiqc rapport voor
het trimmen van de data. Wordt de data beter? Wordt het slechter? Wat
zegt dat over de data? Ik heb eerst de settings gebruikt zoals gelogd op
23-09-2024.\\

Na het trimmen van de data komen er 4 files als output namelijk: -
R1\_paired - R1\_unpaired - R2\_paired - R2\_unpaired

De paired bestanden worden gebruikt voor de verdere bewerking omdat
paired data beide kanten van de read gebruikt. Dit is gunstig omdat je
op deze manier zoveel mogelijk data meeneemt in je beoordeling. Bij
unpaired data wordt maar 1 kant van de read gebruikt waardoor er veel
verlies is van data en dus mogelijk de mutaties die je nodig bent.

Vervolgens heb ik op de paired fastq files fastqc losgelaten en hier een
multiqc van gemaakt.

Eerst fastqc maken van de paired data

\begin{Shaded}
\begin{Highlighting}[]
\ExtensionTok{fastqc}\NormalTok{ /students/2024{-}2025/Thema05/BlaasKanker/oefenengroep/testdataJanine/trimdata/paird/}\PreprocessorTok{*} \DataTypeTok{\textbackslash{}}
    \AttributeTok{{-}o}\NormalTok{ /students/2024{-}2025/Thema05/BlaasKanker/oefenengroep/testdataJanine/trimdata/trimfastqc }\AttributeTok{{-}t}\NormalTok{ 12}
\end{Highlighting}
\end{Shaded}

Daarna multiqc maken van de fastqc's van de paired data

\begin{Shaded}
\begin{Highlighting}[]
\ExtensionTok{multiqc}\NormalTok{ /students/2024{-}2025/Thema05/BlaasKanker/oefenengroep/testdataJanine/trimdata/trimfastqc/}\PreprocessorTok{*} \DataTypeTok{\textbackslash{}}
    \AttributeTok{{-}o}\NormalTok{ /students/2024{-}2025/Thema05/BlaasKanker/oefenengroep/testdataJanine/trimdata/multiqctrim/}
\end{Highlighting}
\end{Shaded}

Door onderstaande settings te gebruiken heb ik een multiqc rapport
gecreeërd, tot onze grote verbazing blijkt deze leeg te zijn. Voor
vandaag heb ik het opgegeven om er nog verder mee bezig te gaan. Ronald
is op de hoogte gesteld en gaat voor ons kijken of hij het kan oplossen.
Uiteindelijk blijkt het probleem alleen bij mij voor te komen, de reden
hiervoor is onbekend en in overleg met Ronald laten we het hierbij voor
dit project.\\

\begin{longtable}[]{@{}
  >{\raggedright\arraybackslash}p{(\columnwidth - 16\tabcolsep) * \real{0.1111}}
  >{\raggedright\arraybackslash}p{(\columnwidth - 16\tabcolsep) * \real{0.1111}}
  >{\raggedright\arraybackslash}p{(\columnwidth - 16\tabcolsep) * \real{0.1111}}
  >{\raggedright\arraybackslash}p{(\columnwidth - 16\tabcolsep) * \real{0.1111}}
  >{\raggedright\arraybackslash}p{(\columnwidth - 16\tabcolsep) * \real{0.1111}}
  >{\raggedright\arraybackslash}p{(\columnwidth - 16\tabcolsep) * \real{0.1111}}
  >{\raggedright\arraybackslash}p{(\columnwidth - 16\tabcolsep) * \real{0.1111}}
  >{\raggedright\arraybackslash}p{(\columnwidth - 16\tabcolsep) * \real{0.1111}}
  >{\raggedright\arraybackslash}p{(\columnwidth - 16\tabcolsep) * \real{0.1111}}@{}}
\toprule\noalign{}
\begin{minipage}[b]{\linewidth}\raggedright
Run ID
\end{minipage} & \begin{minipage}[b]{\linewidth}\raggedright
LEADING
\end{minipage} & \begin{minipage}[b]{\linewidth}\raggedright
TRAILING
\end{minipage} & \begin{minipage}[b]{\linewidth}\raggedright
SLIDING WINDOW
\end{minipage} & \begin{minipage}[b]{\linewidth}\raggedright
MINLEN
\end{minipage} & \begin{minipage}[b]{\linewidth}\raggedright
AVGQUAL
\end{minipage} & \begin{minipage}[b]{\linewidth}\raggedright
HEADCROP
\end{minipage} & \begin{minipage}[b]{\linewidth}\raggedright
CROP
\end{minipage} & \begin{minipage}[b]{\linewidth}\raggedright
ILLUMINACLIP
\end{minipage} \\
\midrule\noalign{}
\endhead
\bottomrule\noalign{}
\endlastfoot
Data van & 3 & 3 & 4:20 & 40 & NONE & NONE & NONE & NONE \\
project & & & & & & & & \\
& & & & & & & & \\
\end{longtable}

Jasper is begonnen met de tool BWA voor de mapping van onze data. Kijk
voor informatie hierover in het logboek van Jasper. Te vinden op onze
github onder logbooks, JasperJonkerLogbooks:
\href{https://github.com/RamonReilman/GenomicsTranscriptomics}{github}\\

\#\#26-09-2024\\
Ramon heeft de volgende trimmomatic settings gebruikt. Dit zijn de
settings die uiteindelijk gebruikt voor de mapping:\\

\begin{longtable}[]{@{}
  >{\raggedright\arraybackslash}p{(\columnwidth - 16\tabcolsep) * \real{0.1111}}
  >{\raggedright\arraybackslash}p{(\columnwidth - 16\tabcolsep) * \real{0.1111}}
  >{\raggedright\arraybackslash}p{(\columnwidth - 16\tabcolsep) * \real{0.1111}}
  >{\raggedright\arraybackslash}p{(\columnwidth - 16\tabcolsep) * \real{0.1111}}
  >{\raggedright\arraybackslash}p{(\columnwidth - 16\tabcolsep) * \real{0.1111}}
  >{\raggedright\arraybackslash}p{(\columnwidth - 16\tabcolsep) * \real{0.1111}}
  >{\raggedright\arraybackslash}p{(\columnwidth - 16\tabcolsep) * \real{0.1111}}
  >{\raggedright\arraybackslash}p{(\columnwidth - 16\tabcolsep) * \real{0.1111}}
  >{\raggedright\arraybackslash}p{(\columnwidth - 16\tabcolsep) * \real{0.1111}}@{}}
\toprule\noalign{}
\begin{minipage}[b]{\linewidth}\raggedright
Run ID
\end{minipage} & \begin{minipage}[b]{\linewidth}\raggedright
LEADING
\end{minipage} & \begin{minipage}[b]{\linewidth}\raggedright
TRAILING
\end{minipage} & \begin{minipage}[b]{\linewidth}\raggedright
SLIDING WINDOW
\end{minipage} & \begin{minipage}[b]{\linewidth}\raggedright
MINLEN
\end{minipage} & \begin{minipage}[b]{\linewidth}\raggedright
AVGQUAL
\end{minipage} & \begin{minipage}[b]{\linewidth}\raggedright
HEADCROP
\end{minipage} & \begin{minipage}[b]{\linewidth}\raggedright
CROP
\end{minipage} & \begin{minipage}[b]{\linewidth}\raggedright
ILLUMINACLIP
\end{minipage} \\
\midrule\noalign{}
\endhead
\bottomrule\noalign{}
\endlastfoot
Data van & 34 & 34 & NONE & 40 & NONE & NONE & NONE & NONE \\
project & & & & & & & & \\
& & & & & & & & \\
\end{longtable}

Ik heb vandaag besteed aan mij inlezen over de tool SnpEff (versie
5.2C), ik wil begrijpen hoe deze tool werkt, waarom we hem gebruiken en
hoe we deze gaan inzetten op ons project. Ik heb de tool alvast op de
schoolserver in ons project gedownload. Ik ga morgen verder met
uitzoeken hoe de tool werkt en meer over de tool leren.

\#\#27-09-2024\\
Ik ben begonnen met het nalezen en aanvullen van mijn logboek. Daarna
ben ik verder gegaan met het leren van SnpEff. De tool is correct
gedownload op de server en werkt. Door Ramon is er een test data set
gemaakt met output van BWA mapping, die ik kan gebruiken om een goede
commandline op te stellen voor deze tool. Dit is handig om dit vooraf te
doen zodat ik, wanneer we eraan toe zijn met de echte data, alleen de
data hoef te veranderen waardoor het gehele groepsproces sneller gaat.
Ik heb dit filmpje gekeken:
\href{https://www.youtube.com/watch?v=-rmreyRAbkE&t=544s}{filmpje
snpeff} en de officiële documentatie gelezen:
\href{https://pcingola.github.io/SnpEff/snpeff/introduction/}{snpEff
documentatie}\\

Iris heeft picard op de echte data uitgevoerd om het hoge aantal
duplicaten te verwijderen. Kijk hiervoor in
\href{https://github.com/RamonReilman/GenomicsTranscriptomics}{Logboek
Iris}\\

\#\#29-09-2024\\
Aan het testen gegaan met snpEff op een testdata set. Deze testdata set
is door Ramon gemaakt en te vinden in:
/students/2024-2025/Thema05/BlaasKanker/testingtools/test\_variant\_calling\\

Test command 1:\\

\begin{Shaded}
\begin{Highlighting}[]
\ExtensionTok{java} \AttributeTok{{-}Xmx8g} \AttributeTok{{-}jar}\NormalTok{ /students/2024{-}2025/Thema05/BlaasKanker/tools/snpeff/snpEff/snpEff.jar eff }\DataTypeTok{\textbackslash{}}
\NormalTok{mm39 /students/2024{-}2025/Thema05/BlaasKanker/testingtools/test\_variant\_calling/SRR14870694\_variants.vcf }\DataTypeTok{\textbackslash{}}
\OperatorTok{\textgreater{}}\NormalTok{ /students/2024{-}2025/Thema05/BlaasKanker/testingtools/test\_variant\_calling/test\_output\_variants.vcf}
\end{Highlighting}
\end{Shaded}

\href{\%5BNotes:\%5D(Notes:)\%7B.uri\%7D}{{[}Notes:\textbackslash\textbackslash{]}(Notes:)\{.uri\}}
- Het command moet uitgevoerd worden vanuit de map waar de tool in
staat.\\
- Er moet een reference genome in het command worden meegenomen, deze
reference genomes zitten in de database van de tool of kunnen ingevoegd
worden. Ons reference genome van de muis is GRCm39.\\

Bij het downloaden van de referencegenome is de volgende foutmelding
gekomen:\\
FATAL ERROR: Failed to download database from
{[}\url{https://snpeff.blob.core.windows.net/databases/v5_2/snpEff_v5_2_mm39.zip},
\url{https://snpeff.blob.core.windows.net/databases/v5_0/snpEff_v5_0_mm39.zip},
\url{https://snpeff.blob.core.windows.net/databases/v5_1/snpEff_v5_1_mm39.zip}{]}\\

Oplossing:\\
Door te zoeken op internet ben ik een github forum tegengekomen die een
mogelijke oplossing bied. Ik ga de volgende stappen uitvoeren vanaf deze
\href{https://github.com/pcingola/SnpEff/issues/536}{github}\\
De stappen beschreven door gebruiker `lukedow':\\

\textbf{Problemsolving}\\
I had the same issue with mm39, but was eventually able to build the db
manually. I tried a few of the options in the snpEff documentation, but
the one that ultimately worked was the .gtf approach.

Most of what you need to know is in the documentation, but not always
super clear. Here were the key points for me:

Make sure you retrieve all of the required FASTA and GTF (and/or GFF)
files from the same genome build, e.g.~from UCSC
(\url{https://useast.ensembl.org/Mus_musculus/Info/Index}). Unzip them
into a specific directory (snpEff/data/mm39/) so the script can find
them, and rename the files:\\
Mus\_musculus.GRCm39.dna.primary\_assembly.fa.gz: sequences.fa
Mus\_musculus.GRCm39.cds.all.fa.gz: cds.fa
Mus\_musculus.GRCm39.pep.all.fa.gz: protein.fa
Mus\_musculus.GRCm39.112.gtf.gz: genes.gtf

Modify the snpEff.config file to include the line: mm39.genome : Mouse

Build the database: java -Xmx4g -jar snpEff.jar build -gtf22 -v mm39

This should create and save the .bin files required for annotation into
/snpEff/data/mm39/ Should be good to go!\\

\#\#Vervolg logboek 29-09-2024\\
Ik heb alle stappen exact zoals hierboven beschreven uitgevoerd. Daarna
heb ik opnieuw onderstaand command uitgevoerd.\\

\begin{Shaded}
\begin{Highlighting}[]
\ExtensionTok{java} \AttributeTok{{-}Xmx8g} \AttributeTok{{-}jar}\NormalTok{ /students/2024{-}2025/Thema05/BlaasKanker/tools/snpeff/snpEff/snpEff.jar eff }\DataTypeTok{\textbackslash{}}
\NormalTok{mm39 /students/2024{-}2025/Thema05/BlaasKanker/testingtools/test\_variant\_calling/SRR14870694\_variants.vcf }\DataTypeTok{\textbackslash{}}
\OperatorTok{\textgreater{}}\NormalTok{ /students/2024{-}2025/Thema05/BlaasKanker/testingtools/test\_variant\_calling/test\_output\_variants.vcf}
\end{Highlighting}
\end{Shaded}

Het command werkt er is een output gecreeërd van een enkel file. Ik ga
nu het command zo aanpassen dat hij de hele map aan vcf files in 1 keer
meeneemt.\\

\begin{Shaded}
\begin{Highlighting}[]
\FunctionTok{ls}\NormalTok{ /students/2024{-}2025/Thema05/BlaasKanker/testingtools/test\_variant\_calling/}\PreprocessorTok{*}\NormalTok{.vcf }\KeywordTok{|} \ExtensionTok{parallel} \AttributeTok{{-}j}\NormalTok{ 4 }\DataTypeTok{\textbackslash{}}
\StringTok{"java {-}Xmx8g {-}jar /students/2024{-}2025/Thema05/BlaasKanker/tools/snpeff/snpEff/snpEff.jar eff mm39 \textgreater{} /students/2024{-}2025/Thema05/BlaasKanker/testingtools/test\_variant\_calling/test\_output\_variants/\{/.\}.vcf"}
\end{Highlighting}
\end{Shaded}

Bovenstaande commandline werkt. Ik voer alle bestanden in vanuit 1 map
en voer met de tool snpEff vervolgens de variant calling uit. In
bovenstaand command gebruik ik alleen de optie `eff', deze optie
annoteert alleen de varianten. Voor de volgende ronde voeg ik de
volgende opties toe:\\

\begin{itemize}
\tightlist
\item
  Xmx8g -\textgreater{} stelt de maximale hoeveelheid geheugen in die
  Java mag gebruiken.\\
\item
  eff -\textgreater{} voor het annoteren van de varianten\\
\item
  verbose (-v) -\textgreater{} Hiermee wordt een gedetailleerd
  logbestand gemaakt die handig kan zijn voor foutopsporing\\
\end{itemize}

Het command ziet er dan alsvolgt uit:\\

\begin{Shaded}
\begin{Highlighting}[]
\FunctionTok{ls}\NormalTok{ /students/2024{-}2025/Thema05/BlaasKanker/testingtools/test\_variant\_calling/}\PreprocessorTok{*}\NormalTok{.vcf }\KeywordTok{|} \ExtensionTok{parallel} \AttributeTok{{-}j}\NormalTok{ 4 }\DataTypeTok{\textbackslash{}}
\StringTok{"java {-}Xmx8g {-}jar /students/2024{-}2025/Thema05/BlaasKanker/tools/snpeff/snpEff/snpEff.jar ann {-}v {-}stats /students/2024{-}2025/Thema05/BlaasKanker/testingtools/test\_variant\_calling/test\_output\_variants/ex1.html mm39 \textgreater{} /students/2024{-}2025/Thema05/BlaasKanker/testingtools/test\_variant\_calling/test\_output\_variants/\{/.\}.vcf"}
\end{Highlighting}
\end{Shaded}

Bovenstaand command werkt, echter is de output leeg van de testdata
leeg. De reden hiervoor is op dit moment onbekend.\\

\#\#01-10-2024\\
We hebben besloten om tussen de stappen variant calling en variant
annotation een filter stap toe te voegen. Dit doen we met de tool
VcfFilter. Deze tool filtert de data dusdanig dat alleen de meest
relevante data behouden wordt. Dit scheelt tijd omdat de hoeveelheid
data aanzienlijk vermindert door de filterstap. Voor de uitvoering kan
er gekeken worden in het logboek van Ramon Reilman. --\textgreater{}\\

Ik ga de gefilterde data gebruiken voor de variant annotation. Ik
gebruik het command zoals ik gister heb vastgesteld. Ik voeg hier alleen
nog de optie -t 8 (8 threads) aan toe zodat er maximaal 8 cpu threads
worden gebruikt. Hiermee reduceer ik de cpu belasting van de
schoolserver. Ik heb dit eerst nog getest op de testdata. Tijdens het
testen ben ik erachter gekomen dat de threads functie niet meer werkend
is. Via dit bericht
\href{https://github.com/pcingola/SnpEff/issues/429}{github issues} is
te lezen dat het een bekend probleem is. Echter is de documentatie van
snpEff nog niet aangepast. Ik ga nu gebruik maken van de parallel -j
functie om de threads te verdelen. Ik gebruik parallel -j 8 om zo de
workload op de schoolserver te verminderen.\\

\begin{Shaded}
\begin{Highlighting}[]
\FunctionTok{ls}\NormalTok{ /students/2024{-}2025/Thema05/BlaasKanker/outputs/variant\_calling\_unfiltered/}\PreprocessorTok{*}\NormalTok{.vcf }\KeywordTok{|} \ExtensionTok{parallel} \AttributeTok{{-}j}\NormalTok{ 8 }\DataTypeTok{\textbackslash{}}
\StringTok{"java {-}Xmx8g {-}jar /students/2024{-}2025/Thema05/BlaasKanker/tools/snpeff/snpEff/snpEff.jar ann {-}v {-}stats /students/2024{-}2025/Thema05/BlaasKanker/outputs/variant\_annotation\_unfiltered/stats.html mm39 \textgreater{} /students/2024{-}2025/Thema05/BlaasKanker/outputs/variant\_annotation\_unfiltered/\{/.\}.vcf"}
\end{Highlighting}
\end{Shaded}

Na het uitvoeren van bovenstaand command krijg ik de volgende
foutmelding: WARNING!: Mitochondrion chromosome `MT' does not have a
mitochondrion codon table (codon table = `Standard'). You should update
the config file. Ik probeer uit te zoeken waar ik de tabel kan vinden om
deze in het configuration file toe te voegen.\\

Omdat ik niet precies weet waar het misgaat heb ik besloten de database
rondom mm39 opnieuw te bouwen. Hiervoor volg ik deze instructie
{[}github instructie{]}
(\url{https://pcingola.github.io/SnpEff/snpeff/build_db/})\\
Ik heb van deze link op het
\href{https://www.ncbi.nlm.nih.gov/datasets/genome/GCF_000001635.27/}{ncbi}
de fasta sequence, gft en gff file gedownload. Deze heb ik in de
volgende map gezet: ~
/students/2024-2025/Thema05/BlaasKanker/tools/snpeff/snpEff/data/mm39.\\

Na het opnieuw bouwen van de database en uitvoeren van de tool op onze
data krijg ik nog steeds de foutmelding: Mitochondrion chromosome `MT'
does not have a mitochondrion codon table (codon table = `Standard').\\

Ik heb voor nu besloten om hulp te vragen bij de docenten, ook Ramon
kijkt nog even mee.~

Ramon heeft Lofreq uitgevoerd voor de variant calling. Te vinden op onze
\href{https://github.com/RamonReilman/GenomicsTranscriptomics}{github}
onder logbooks, RamonReilmanLogbooks:\\

\#\#2-10-2024\\
Ramon heeft laten zien dat hij door middel van nog 2 andere tools mijn
tool snpEff bijna werkend heeft gekregen. Kijk voor de stappen die hij
heeft ondernomen in zijn logboek: Te vinden op onze
\href{https://github.com/RamonReilman/GenomicsTranscriptomics}{github}
onder logbooks, RamonReilmanLogbooks:\\

\#\#3-10-2024\\
Ik heb afgelopen week feedback gekregen op mijn logboek van Ronald.
Hierbij is gebleken dat ik alles veel duidelijker en uitgebreider moet
verwoorden. Daarom neem ik hier vandaag de tijd voor om het hele logboek
van vooraan tot en met nu volledig aan te vullen. Hierbij ga ik de hoe,
wat en waarom toepassen om op deze manier zoveel mogelijk
tijdsverantwoording te vangen.\\

Ik ga onderzoeken of we de output vcf bestanden die uit de variant
annotation komen ook kunnen filteren op impact van mutaties. Hiermee
zouden we het lezen en interpreteren van de mutaties namelijk veel
makkelijker maken. Ik ga kijken of hier tools voor zijn en en welke dat
zijn. ~

Er zijn in principe 2 tools die dit kunnen doen en dit zijn VEP (Variant
effect predictor) en SnpSift. Omdat SnpSift samenwerkt met SnpEff is het
in dit geval de meest gunstige tool om te gebruiken. SnpSift is een
toolbox met heel veel verschillende opties. Ik ga onderzoeken welke
opties we het beste kunnen gebruiken om een zo gunstig mogelijke
uitkomst te genereren. Wanneer de meest optimale uitkomst hebben kunnen
we hier makkelijker plotjes van maken voor ons wetenschappelijke
artikel.

\#\#04-10-2024\\
Na overleg met de groep hebben we besloten om SnpSift te gaan gebruiken
om de impact van de genen en de soort puntmutatie eruit te filteren.
SnpSift heeft hier een filter functie voor met de opties IMPACT en
FUNCLASS.\\
In de IMPACT optie kunnen de volgende waarden meegegeven worden:\\
- High\\
- Moderate\\
- Low\\
- Modifier ~

In de FUNCLASS optie kunnen de volgende waarden meegegeven worden:\\
- None\\
- Silent\\
- Missense\\
- Nonsense\\

De impact wat de varianten van genen hebben kan betekenen dat het
verloop van de ziekte, hoe erg het zich uit bijvoorbeeld, daarmee aan te
duiden is.\\

Wanneer we het over de modifier optie hebben, dan gaat het om modifier
genes. Deze genen spelen een rol in de variabiliteit van de fenotypische
expressie van de ziekte. In het geval van blaaskanker bij muizen, kunnen
modifier-genen de groei, progressie en respons op behandeling van
tumoren beïnvloeden. Deze genen kunnen varianten bevatten die de
expressie van andere genen moduleren, wat leidt tot verschillen in
tumorontwikkeling en -gedrag tussen individuen. Dit is dus een
belangrijke optie om mee te geven, deze genen willen we wel meenemen in
het onderzoek.\\

Low impact genen (optie low impact) zijn meestal varianten die geen tot
weinig impact hebben op de ziekte en het verloop hiervan. In
\href{https://genomebiology.biomedcentral.com/articles/10.1186/s13059-017-1212-4\#Sec1}{dit
artikel} is wel te lezen dat er nog veel onderzoek is te doen naar low
impact genen omdat het wel de fenotypische expressie kan beïnvloeden.
Echter is er nog zeer weinig bekend over wat nu de precieze impact is
van low impact genes. Voor ons onderzoek lijkt het niet relevant te zijn
om deze genen mee te nemen dus dat gaan we niet doen.\\

High impact genes hebben zoals de naam al doet vermoeden een zeer grote
impact op de ziekte en het verloop hiervan. Deze genen kunnen
aanzienlijke veranderingen veroorzaken in eiwitten of de regulatie van
de genen. Door deze veranderingen hebben ze een zeer groot fenotypisch
effect hebben op de expressie en verloop van de ziekte. Deze optie wordt
dus meegenomen in ons onderzoek.\\

Tijdens het onderzoeken naar al deze achtergrondinformatie hebben we
besloten de tool toch niet te gaan gebruiken. Daarom ga ik geen extra
tijd besteden aan meer theoretische achtergrond zoeken.\\

\#\#11-10-2024\\
Ramon is vandaag nog met de plotjes bezig om te kijken of dit werkend te
krijgen is. Jasper, Iris en ik hebben de logboeken tegen elkaar aan
gelegd en alle belangrijke stappen overgenomen van elkaar. Ook heb ik
nog het logboek aangevuld met informatie.

\#\#16-10-2024\\
Ik heb de lollipop plot van
\href{https://github.com/RamonReilman/GenomicsTranscriptomics/tree/main/logbooks}{Ramon
zijn logboek} overgenomen.\\
\includegraphics{/homes/jwpostmus/Documents/Variants_on_Gstm3.png}

\end{document}
